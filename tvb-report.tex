%----------------------------------------------------------------------------------------
%	FONTS
%----------------------------------------------------------------------------------------

\documentclass[10pt,parskip=half]{scrartcl}

%----------------------------------------------------------------------------------------
%	PACKAGES AND OTHER DOCUMENT CONFIGURATIONS
%----------------------------------------------------------------------------------------

\usepackage{amsmath, amsfonts, amsthm} % Math packages
\usepackage{listings} % Code listings, with syntax highlighting
\usepackage[english]{babel} % English language hyphenation
\usepackage{graphicx} % Required for inserting images
\usepackage{booktabs} % Required for better horizontal rules in tables

\usepackage{float}   % for [H]
\usepackage{needspace} % to ensure enough room for heading+figure

\numberwithin{equation}{section} % Number equations within sections
\numberwithin{figure}{section}   % Number figures within sections
\numberwithin{table}{section}    % Number tables within sections

\setlength\parindent{0pt} % Removes all indentation from paragraphs

\usepackage{enumitem} % Required for list customisation
\setlist{noitemsep} % No spacing between list items

\usepackage{hyperref} % For hyperlinks in the PDF

%----------------------------------------------------------------------------------------
%	DOCUMENT MARGINS
%----------------------------------------------------------------------------------------

\usepackage{geometry} % Required for adjusting page dimensions and margins

\geometry{
	paper=a4paper,
	top=1.5cm,
	bottom=2cm,
	left=2cm,
	right=2cm,
	headheight=0.75cm,
	footskip=1.5cm,
	headsep=0.75cm,
	%showframe,
}

%----------------------------------------------------------------------------------------
%	SECTION TITLES (REPLACEMENT FOR sectsty)
%----------------------------------------------------------------------------------------

% --- SECTION ---
\setkomafont{section}{\normalfont\large\bfseries}
%\setcounter{secnumdepth}{0}
\renewcommand*\sectionformat{} % hide number only in the section heading
% --- SUBSECTION ---
\setkomafont{subsection}{\normalfont\bfseries}
% --- SUBSUBSECTION ---
\setkomafont{subsubsection}{\normalfont\itshape}
% --- PARAGRAPH ---
\setkomafont{paragraph}{\normalfont\scshape}

%----------------------------------------------------------------------------------------
%	HEADERS AND FOOTERS
%----------------------------------------------------------------------------------------

\usepackage{scrlayer-scrpage} % Required for customising headers and footers

\ohead*{} % Right header
\ihead*{} % Left header
\chead*{} % Centre header

\ofoot*{} % Right footer
\ifoot*{} % Left footer
\cfoot*{\pagemark} % Centre footer

%----------------------------------------------------------------------------------------
%	TITLE SECTION
%----------------------------------------------------------------------------------------

\title{	
	\normalfont\normalsize
	\vspace{-15pt}
	%\textsc{Vrije Universiteit Amsterdam - School of Business and Economics}\\
	%\textsc{Econometrics for Quantitative Finance}\\
	%\vspace{25pt} % Whitespace
	%\rule{\linewidth}{0.5pt}\\ % Thin top horizontal rule
	%\vspace{20pt} % Whitespace
	{\huge Estimating Time-Varying Betas for Stock Returns}\\ % The assignment title
	\vspace{10pt} % Whitespace
	{\huge DCC-GARCH Method}\\ % The assignment title
	%\vspace{12pt} % Whitespace
	%\rule{\linewidth}{2pt}\\ % Thick bottom horizontal rule
	\vspace{0pt} % Whitespace
}

\author{
    \LARGE Milan Peter \\
    \large September 2025 \\
}

\date{}

\begin{document}

\maketitle % Print the title
\vspace{-3.5em}  % reduce vertical space between title and text

%----------------------------------------------------------------------------------------
%	INTRO
%----------------------------------------------------------------------------------------
\begin{quote}
This report was created as a practical task related to the project ``Estimating Dynamic Exposures of Banks for Cyber Risks'' at Vrije Universiteit Amsterdam.
My task was to take any publicly available daily factor and a small panel of stock returns or simulated data, and estimate the time-varying betas to the selected factor using a freely selected method.
Finally, I was to include a figure with all beta estimates + 95\% CI.
The report summarizes my approach, key findings, and any challenges.
\end{quote}
%----------------------------------------------------------------------------------------
%	SECTION
%----------------------------------------------------------------------------------------

\section{Theoretical Background}
\vspace{-1em}
In a standard CAPM regression setting, a stock's excess return is modelled as

\[
r_{i,t} = \alpha_i + \beta_i f_t + \varepsilon_{i,t}, \tag{1}
\]

where the \(r_{i,t}\) is the return of stock \(i\) at time \(t\), \(f_t\) is the return of factors (e.g., market, SMB, or cyber-risk index in our case), and \(\beta_i\) is the exposure/sensitivity of stock \(i\) to factor \(f\).
Usually, \(\beta_i\) is assumed constant over the sample.
But exposures shift: banks may become more sensitive to cyber-risk factors after a major incident, then less sensitive later, for example. So, we allow \(\beta_{i,t}\) to change over time:

\[
r_{i,t} = \alpha_i + \beta_{i,t} f_t + \varepsilon_{i,t}, \tag{2}
\]

where \(\beta_{i,t}\) is the time-varying beta.

There are several common approaches to estimate time-varying betas: rolling window OLS, DCC-GARCH and Kalman filter.
In this short report, I will focus on DCC-GARCH (Dynamic Conditional Correlation - Generalized Autoregressive Conditional Heteroskedasticity) that models the time-varying covariance matrix of multiple return series.
If we fit a multivariate DCC-GARCH on stock \(r_{i,t}\) and factor \(f_t\), we get conditional variance of factor: \(\operatorname{Var}_t(f)\), and conditional covariance between stock and factor: \(\operatorname{Cov}_t(r_i,f)\). Then

\[
\beta_{i,t} = \frac{\operatorname{Cov}_t(r_i,f)}{\operatorname{Var}_t(f)} \tag{3} \label{eq:time-varying-beta}
\]

So, the DCC model naturally produces dynamic betas.

%-------------------------------------------

\section{Data}
\vspace{-1em}

As for the factor, I use the market (MKT) factor from the Fama-French 3-Factor model.
This is used as a proxy for the overall market return.
Using a factor related to cyber-risk would be more relevant to the context of the project, but such data is not easy to obtain.
Thus, I proceed with the market factor, which can be accessed through the \href{https://mba.tuck.dartmouth.edu/pages/faculty/ken.french/Data_Library/f-f_factors.html}{Fama-French} website.
As for the panel of stock, given the context of the project, I consider six banking stocks from the S\&P 500 index.
The stock data is obtained using the WRDS database, specifically the \href{https://wrds-www.wharton.upenn.edu/pages/about/data-vendors/center-for-research-in-security-prices-crsp/\#products}{CRSP} dataset, as the university has a subscription to it.
The data spans from June 30, 2015, to June 30, 2025, covering a period of roughly ten years.
The stocks selected are: Citigroup (C), JPMorgan Chase (JPM), Bank of America (BAC), Wells Fargo (WFC), Goldman Sachs (GS).
The factor used is: market factor \(R_m - R_f\) (denoted as MKT).

The data files are included in the repository as \texttt{data\_ff3.csv} and \texttt{data\_prices.csv}.
After importing the data, I calculate the daily returns for each stock and merge the stock returns with the factor data into a single panel DataFrame.
The returns are multiplied by 100 to express them in percentage terms, and to ensure consistency with the factor returns, which are also in percentage terms.
The resulting DataFrame has a multi-index with dates and stock tickers, and columns for stock returns and factor returns.

%-------------------------------------------

\section{Estimation}
\vspace{-1em}

I start with defining the parameters.
I work with an estimation window of 252 trading days (approximately one year), and confidence intervals are computed at the 95\% level, as requested.
To estimate the time-varying betas, I loop over each stock in the panel.
For each stock, I extract its returns and the factor returns, and fit a GARCH(1,1) model using the \texttt{arch} library.
The GARCH model captures the dynamic correlations between the stock and the factor over time.
After fitting the model, I extract the conditional covariance between the stock and the factor, as well as the conditional variance of the factor.
Using Equation \ref{eq:time-varying-beta}, I compute the time-varying beta for each stock.
I use the standard errors from the GARCH model to compute the 95\% confidence intervals for the betas.
For comparison, I also compute the OLS beta using the same returns using the \texttt{statsmodels} library.
As a final step, I store the results in a dictionary for later use.

%-------------------------------------------

\section{Results}
\vspace{-1em}

The results in Table \ref{tab:comparison-table} and Figure \ref{fig:time-varying-betas} highlight clear evidence that the systematic risk exposures of large U.S. banks fluctuate considerably over time.
The constant-beta estimates from the standard OLS CAPM regression give a baseline of each firm's average sensitivity to market returns.
However, the DCC-GARCH results reveal that these exposures are far from constant.

Across all six institutions, the mean of the time-varying betas exceeds the static OLS beta, suggesting that the traditional constant-beta model slightly underestimates the degree of market sensitivity once time variation is accounted for.
For example, Citigroup (C) shows an OLS beta of 1.29, while its average dynamic beta is 1.35, with a standard deviation of 0.47 – indicating frequent and large deviations from the long-term mean.

The range column in Table \ref{tab:comparison-table} shows substantial variation in market exposure across time.
Citigroup (C) and Bank of New York Mellon (BK) have the widest ranges, suggesting that their sensitivity to market risk occasionally tripled during volatile periods.
These peaks align with systemic stress episodes such as the 2020 pandemic and the 2022 inflation shocks.

Figure \ref{fig:time-varying-betas} confirms these dynamics: betas rise sharply during crises and settle afterward, while confidence intervals widen when volatility increases.
Goldman Sachs (GS) and Bank of America (BAC) display comparatively stable betas post-2021, whereas Citigroup (C) and Wells Fargo (WFC) fluctuate more strongly.
JPMorgan Chase (JPM) shows the most stable pattern, reflecting steadier market exposure.

\begin{table}[htbp]
\centering
\caption{Comparison of Beta Estimates across Firms}
\label{tab:comparison-table}
\makebox[\textwidth]{
  \input{output/comparison_table.tex}
}
\end{table}

From a methodological standpoint, the results confirm that the DCC-GARCH framework is capable of revealing both firm-specific and system-wide risk dynamics hidden in constant-parameter models.
The time-varying betas can serve as inputs for further analysis – such as stress testing, portfolio allocation, or risk management applications – by reflecting the evolving co-movement between bank stocks and the market factor.

Future work could extend this analysis by introducing additional factors (e.g. a sophisticated cyber-risk index) or by benchmarking the DCC-GARCH results against alternative time-variation estimators like the Kalman filter.
Such comparison would clarify whether observed fluctuations stem from genuine economic shifts or model-driven volatility in the conditional correlations.

%----------------------------------------------------------------------------------------
%	REFERENCES
%----------------------------------------------------------------------------------------

\begin{thebibliography}{9}

\bibitem{fama1993}
Fama, E. F. and K. R. French (1993). Common Risk Factors in the Returns on Stocks and Bonds. \textit{Journal of Financial Economics} 33.1, pp. 3–56.

\bibitem{engle2002}
Engle, R. (2002). Dynamic Conditional Correlation: A Simple Class of Multivariate GARCH Models. \textit{Journal of Business \& Economic Statistics} 20.3, pp. 339–350.

\end{thebibliography}

%----------------------------------------------------------------------------------------
%	APPENDIX
%----------------------------------------------------------------------------------------

\clearpage

\setcounter{section}{0}
\renewcommand\thesection{\alph{section}}
\section{Appendix}

\begin{figure}[H]
    \centering
    \caption{Time-Varying Factor Betas for Banking Stocks}
    \includegraphics[width=\textwidth]{output/time_varying_betas.png}
    \label{fig:time-varying-betas}
\end{figure}

%----------------------------------------------------------------------------------------
%	END OF DOCUMENT
%----------------------------------------------------------------------------------------

\end{document}
